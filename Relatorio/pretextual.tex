%!TEX root = ../Principal.tex
% Capa do trabalho
\imprimircapa


% Folha de rosto
%* indica que tem ficha catalográfica
\imprimirfolhaderosto*


% Caso a Biblioteca da UDESC forneça, utilize o comando
% \begin{fichacatalografica}
%     \includepdf{fig_ficha_catalografica.pdf}
% \end{fichacatalografica}

% Geração da ficha catalográfica via LaTeX
\begin{fichacatalografica}
	\vspace*{\fill}					% Posição vertical
	\begin{center}					% Minipage Centralizado
	\begin{minipage}[c]{12.5cm}		% Largura
	
	\imprimirautor
	
	\hspace{0.5cm} \imprimirtitulo  / \imprimirautor. --
	\imprimirlocal, \imprimirdata-
	
	\hspace{0.5cm} \pageref{LastPage} p. : il. (algumas color.) ; 30 cm.\\
	
	\hspace{0.5cm} \imprimirorientadorRotulo~\imprimirorientador\\
	
	\hspace{0.5cm}
	\parbox[t]{\textwidth}{\imprimirtipotrabalho~--~\imprimirinstituicao,
	\imprimirdata.}\\
	
	\hspace{0.5cm}
		1. Tópico 01.
		2. Tópico 02.
		I. Prof. Dr. xxxxx.
		II. Universidade do Estado de Santa Catarina.
		III. Centro de Educação do Planalto Norte.
		IV. identificação xxxx\\ 			
	
	\hspace{8.75cm} CDU 02:121:005.7\\
	
	\end{minipage}
	\end{center}
\end{fichacatalografica}


% Folha de aprovação
% Exemplo de folha de aprovação antes da Banca. Após isso, incluia o pdf digitalizado com as assinaturas%
% \includepdf{folhadeaprovacao_final.pdf}
\begin{folhadeaprovacao}

	\begin{center}
		{\ABNTEXchapterfont\bfseries\imprimirautor}
		\vspace{6em}

			\ABNTEXchapterfont\bfseries\imprimirtitulo
		
	\end{center}
		\vspace{1em}
		{\justify
		Trabalho de Conclusão de Curso apresentado ao curso de Bacharelado em Ciência da Computação como requisito parcial para a obtenção do título de Bacharel em Ciência da Computação}
	
	\vspace{3em} 
	\noindent
	{\bfseries Banca examinadora:}
	\assinatura{\textbf{Dra. Karina Girardi Roggia} \\ Universidade do Estado de Santa Catarina (Udesc)} 
	\assinatura{\textbf{Dr. Cristiano Damiani Vasconcellos} \\ Udesc}
    \assinatura{\textbf{Dr. Roberto Silvio Ubertino Rosso Junior} \\ Udesc}

    \vspace*{\fill}
    \begin{center}
    	\imprimirlocal,\,\imprimirfulldata
    \end{center}
\end{folhadeaprovacao}


% Dedicatória
%\begin{dedicatoria}				
%Dedico este trabalho a...  
%\end{dedicatoria}


% Agradecimentos
%\begin{agradecimentos}
%Gostaria de agradecer...

%Aqui devem ser colocadas os agradecimentos às pessoas que de alguma forma contribuíram para a realização do trabalho.
%\end{agradecimentos}


% Epígrafe
%\begin{epigrafe}	
%``frase''
%\\
%\par
%autor
%\end{epigrafe}


% Resumo em português
\begin{resumo}
 O resumo deve ressaltar o
 objetivo, o método, os resultados e as conclusões do documento. A ordem e a extensão
 destes itens dependem do tipo de resumo (informativo ou indicativo) e do
 tratamento que cada item recebe no documento original. O resumo deve ser
 precedido da referência do documento, com exceção do resumo inserido no
 próprio documento. (\ldots) As palavras-chave devem figurar logo abaixo do
 resumo, antecedidas da expressão Palavras-chave:, separadas entre si por
 ponto e finalizadas também por ponto.

 \vspace{\onelineskip}
    
 \noindent
 \textbf{Palavras-chaves}: latex, abntex e editoração de texto.
\end{resumo}


% Resumo em inglês
\begin{resumo}[Abstract]
 \begin{otherlanguage*}{english}
	Resumo em inglês
   \vspace{\onelineskip}
 
   \noindent 
   \textbf{Key-words}: latex, abntex e text editoration.
 \end{otherlanguage*}
\end{resumo}


% Lista de figuras
\pdfbookmark[0]{\listfigurename}{lof}
\listoffigures*
\cleardoublepage


% Lista de tabelas
%\pdfbookmark[0]{\listtablename}{lot}
%\listoftables*
%\cleardoublepage


% Lista de quadros
\pdfbookmark[0]{\listofquadrosname}{loq}
\listofquadros*
\cleardoublepage
