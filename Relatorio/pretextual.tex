%!TEX root = ../Principal.tex
% Capa do trabalho
\imprimircapa


% Folha de rosto
%* indica que tem ficha catalográfica
\imprimirfolhaderosto*


% Caso a Biblioteca da UDESC forneça, utilize o comando
% \begin{fichacatalografica}
%     \includepdf{fig_ficha_catalografica.pdf}
% \end{fichacatalografica}

% Geração da ficha catalográfica via LaTeX
%\begin{fichacatalografica}
%	\vspace*{\fill}					% Posição vertical
%	\begin{center}					% Minipage Centralizado
%	\begin{minipage}[c]{12.5cm}		% Largura
%	
%	\imprimirautor
%	
%	\hspace{0.5cm} \imprimirtitulo  / \imprimirautor. --
%	\imprimirlocal, \imprimirdata-
%	
%	\hspace{0.5cm} \pageref{LastPage} p. : il. (algumas color.) ; 30 cm.\\
%	
%	\hspace{0.5cm} \imprimirorientadorRotulo~\imprimirorientador\\
%	
%	\hspace{0.5cm}
%	\parbox[t]{\textwidth}{\imprimirtipotrabalho~--~\imprimirinstituicao,
%	\imprimirdata.}\\
%	
%	\hspace{0.5cm}
%		1. Tópico 01.
%		2. Tópico 02.
%		I. Prof. Dr. xxxxx.
%		II. Universidade do Estado de Santa Catarina.
%		III. Centro de Educação do Planalto Norte.
%		IV. identificação xxxx\\ 			
%	
%	\hspace{8.75cm} CDU 02:121:005.7\\
%	
%	\end{minipage}
%	\end{center}
%\end{fichacatalografica}


% Folha de aprovação
% Exemplo de folha de aprovação antes da Banca. Após isso, incluia o pdf digitalizado com as assinaturas%
% \includepdf{folhadeaprovacao_final.pdf}
%\begin{folhadeaprovacao}
%
%	\begin{center}
%		{\ABNTEXchapterfont\bfseries\imprimirautor}
%		\vspace{6em}
%
%			\ABNTEXchapterfont\bfseries\imprimirtitulo
%		
%	\end{center}
%		\vspace{1em}
%		{\justify
%		Trabalho de Conclusão de Curso apresentado ao curso de Bacharelado em Ciência da Computação como requisito parcial para a obtenção do título de Bacharel em Ciência da Computação}
%	
%	\vspace{3em} 
%	\noindent
%	{\bfseries Banca examinadora:}
%	\assinatura{\textbf{Dra. Karina Girardi Roggia} \\ Universidade do Estado de Santa Catarina (Udesc)} 
%	\assinatura{\textbf{Dr. Cristiano Damiani Vasconcellos} \\ Udesc}
%    \assinatura{\textbf{Dr. Roberto Silvio Ubertino Rosso Junior} \\ Udesc}
%
%%    \vspace*{\fill}
%    \begin{center}
%    	\imprimirlocal,\,\imprimirfulldata
%    \end{center}
%\end{folhadeaprovacao}


% Dedicatória
%\begin{dedicatoria}				
%Dedico este trabalho a...  
%\end{dedicatoria}


% Agradecimentos
%\begin{agradecimentos}
%Gostaria de agradecer...

%Aqui devem ser colocadas os agradecimentos às pessoas que de alguma forma contribuíram para a realização do trabalho.
%\end{agradecimentos}


% Epígrafe
%\begin{epigrafe}	
%``frase''
%\\
%\par
%autor
%\end{epigrafe}


% Resumo em português
\begin{resumo}
 Os autômatos finitos determinísticos, reconhecedores de linguagens regulares, são muito importantes para a ciência da computação. Na automação industrial, eles podem modelar sistemas orientados a eventos discretos e assíncronos, denominados sistemas a eventos discretos. Nesse contexto, os autômatos são máquinas abstratas que, ao computar uma cadeia de símbolos que representam eventos, descrevem o funcionamento do sistema por meio de estados e transições. Uma das classes de sistemas a eventos discretos é a de sistemas de filas, nos quais há uma fila ou \textit{buffer} e algumas propriedades que devem ser respeitadas no projeto e implementação do sistema. Este trabalho investiga a viabilidade do uso de assistentes de provas a fim de garantir propriedades de sistemas modelados por autômatos finitos determinísticos. Foi adotado o assistente Coq como ferramenta, e serão provadas propriedades do problema produtor-consumidor como estudo de caso.

 \vspace{\onelineskip}
    
 \noindent
 \textbf{Palavras-chaves}: autômatos finitos determinísticos, assistente de provas, sistemas a eventos discretos, sistemas de filas, produtor-consumidor.
\end{resumo}


% Resumo em inglês
\begin{resumo}[Abstract]
 \begin{otherlanguage*}{english}
 Deterministic finite automata, recognizers of regular languages, are deeply relevant in computer science. Regarding to industrial automation, they play a very important role in modeling discrete and asynchronous event-oriented systems, called discrete event systems. In this context, automata are abstract machines that, when computing a string representing a sequence of events, describe the functioning of the system through states and transitions. One of the classes of discrete event systems is formed by queuing systems, where there is a queue or buffer and some properties that must be satisfied by system design and implementation. This paper investigates the feasibility of using proof assistants to assure properties of systems modeled by deterministic finite automata. The Coq assistant has been adopted as a tool, and properties of the producer-consumer problem will be proved as a case study.
 
 \vspace{\onelineskip}
 
 \noindent 
 \textbf{Keywords}: deterministic finite automata, proof assistant, discrete event systems, queueing systems, producer-consumer.
 \end{otherlanguage*}
\end{resumo}


% Lista de figuras
\pdfbookmark[0]{\listfigurename}{lof}
\listoffigures*
\cleardoublepage


% Lista de tabelas
%\pdfbookmark[0]{\listtablename}{lot}
%\listoftables*
%\cleardoublepage


% Lista de quadros
\pdfbookmark[0]{\listofquadrosname}{loq}
\listofquadros*
\cleardoublepage
