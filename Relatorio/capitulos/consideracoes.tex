\chapter{Considerações parciais}
\label{cap:consideracoes}

Este trabalho apresentou os principais conceitos relacionados à modelagem de sistemas a eventos discretos (\acs{SED}s) por meio de autômatos finitos determinísticos (\acs{AFD}s), bem como os relativos a sistemas de filas. Além disso, foi apresentada a possibilidade de reduzir problemas de SEDs à garantia de propriedades acerca de filas abstratas. Demonstraram-se outras aplicações envolvendo duas restrições de sistemas de filas, que podem ser verificadas por meio de teoremas provados com assistência do Coq. Esse assistente de provas, introduzido no capítulo \ref{cap:provas}, mostrou-se útil no processo de prova interativa, posteriormente transcrita para o relatório.

Os exemplos e aplicações expostos neste relatório evidenciam a relevância de provar a consistência de sistemas com filas, cujas falhas decorrentes do projeto resultam em problemas para a produção automatizada. Também é eminente a utilidade dos formalismos matemáticos na verificação de características esperadas para um sistema. Após o SED ser projetado e modelado, é possível realizar estudos a respeito de seu funcionamento, em que a matemática é artifício de importância.

Alguns teoremas expostos nesta monografia não foram devidamente provados, posto que isso é trabalho futuro. Ainda, novos problemas, aplicações, exemplos, lemas e teoremas podem vir à tona, o que será acrescido ao trabalho final.