\chapter{Considerações parciais}
\label{cap:consideracoes}

Este trabalho apresentou os principais conceitos relacionados à modelagem de sistemas a eventos discretos (\acs{SED}s) por meio de autômatos finitos determinísticos (\acs{AFD}s), bem como os relativos a sistemas de filas. As duas restrições do problema do produtor-consumidor foram abordadas, originando métodos formais a fim de garantir as propriedades da capacidade máxima e volume mínimo zero em sistemas de filas. Um teorema foi provado mediante o assistente de provas Coq, enquanto outro será provado na continuação deste trabalho de conclusão de curso.

Como resultado de produção do trabalho, foram escritas 237 linhas de código Gallina na IDE do Coq. O processo passou por uma formalização inicial, que foi simplificada gradativamente. O estado parcial da implementação é uma interface mais bem estruturada e organizada.

Entre as dificuldades encontradas durante a produção deste trabalho, pode-se citar a procura de aplicações realistas de SEDs e sistemas de filas. Os exemplos citados neste trabalho são demasiados simples, não correspondendo a sistemas reais da automação industrial. Outro impasse surgiu nos estudos bibliográficos acerca dos assistentes de provas, uma vez que há muitas teorias que sustentam tais artefatos.

Para a continuidade do presente trabalho, novos teoremas e problemas poderão ser formulados. A generalização dos métodos formais para autômatos finitos não determinísticos também possivelmente será feita, assim como a especificação e verificação de algoritmos de \textit{model-checking} baseados nos teoremas.

Com respeito aos desenvolvimentos feitos no Coq, as formulações e provas relativas aos teoremas expostos neste trabalho encontram-se na página do Github do autor, acessíveis em: $$\text{\href{https://github.com/fil1pe/tcc}{\texttt{https://github.com/fil1pe/tcc}}}$$