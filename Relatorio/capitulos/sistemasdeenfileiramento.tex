\chapter{Sistemas de enfileiramento}

Os sistemas de enfileiramento constituem importante classe de \acs{SED}s. \hl{REVISAR.} \cite{cassandras}

Para ser atendido por um servidor de banco, uma pessoa deve esperar em uma fila até que as pessoas a sua frente sejam atendidas. Esse sistema de enfileiramento pode ser modelado pelo autômato infinito determinístico ilustrado pela Figura \ref{fig:sist_enfil_simples_infinito}, em que $+$ e $-$ são os respectivos eventos de chegada e partida de pessoa da fila.

\figuradoautor{Um sistema de enfileiramento simples}{
	\begin{tikzpicture}[shorten >=1pt,node distance=2cm,on grid,auto] 
		\node[state,initial] (q0) {}; 
		\node[state] at (3,0) (q1) {};
		\node[state] at (6,0) (q2) {};
		\node[state,draw=none] at (9,0) (q3) {$...$};
		\draw
		(q0) edge[bend left, above] node{$+$} (q1)
		(q1) edge[bend left, below] node{$-$} (q0)
		(q1) edge[bend left, above] node{$+$} (q2)
		(q2) edge[bend left, below] node{$-$} (q1)
		(q2) edge[bend left, above] node{$+$} (q3)
		(q3) edge[bend left, below] node{$-$} (q2);
		\end{tikzpicture}
}{fig:sist_enfil_simples_infinito}

Na Figura \ref{fig:sist_enfil_simples_infinito} é assumido que a fila de pessoas pode crescer indefinidamente, o que não é possível na realidade.

Um exemplo de sistema de enfileiramento é apresentado por \citeonline{victor} e consiste em uma planta de fila de demandas para veículos aéreos não tripulados. \hl{PROBLEMA DO VICTOR}

\section{Problema do produtor e consumidor}

\section{Propriedades desejadas}

\subsection{Aplicações}
