\chapter{Garantia de propriedades em sistemas de filas}

\hl{ESCREVER MAIS}

Para representar as operações de adição e retirada de fila, utilizaremos os respectivos símbolos $+$ e $-$. Seja $G = \langle Q, E, \delta, q_0, Q_m \rangle$ um AFD definido conforme a equação \ref{eq:afd}, em que $E \supseteq \{ +, - \}$, queremos provar que o sistema de filas modelado por $G$ satisfaz as duas propriedades supracitadas no capítulo \ref{cap:filas}: que a fila tenha uma capacidade máxima e não seja permitido efetuar operação de retirada de item quando ela estiver vazia.

Precisamos de uma função $c : E^\star \to \mathbb{Z}$ que conte o número de itens armazenados em uma fila após uma cadeia de eventos $w$. Uma forma de defini-la é $$c(w) = \begin{cases}
c(w') + 1 & \text{se $w=+w'$}\\
c(w') - 1 & \text{se $w=-w'$}\\
0 & \text{senão}
\end{cases}$$ Baseado nas propriedades aritméticas da soma de inteiros, podemos observar para quaisquer $w_1$ e $w_2 \in E^\star$ $$c(w_1w_2) = c(w_1) + c(w_2)$$ que será útil para a prova dos teoremas acerca das propriedades dos sistemas de filas.

A fim de facilitar a leitura dos teoremas, simplifiquemos a notação para a transição estendida de um estado $q$ por uma cadeia de eventos $w$ da seguinte maneira: $$qw \equiv \hat{\delta}(q,w)$$ Pelo mesmo motivo, definamos a função $t : E \to \mathbb{Z}$ desta forma: $$t(e) = \begin{cases}
1 & \text{se $e=+$} \\
-1 & \text{se $e=-$} \\
0 & \text{senão}
\end{cases}$$

\section{Garantia de capacidade máxima}

\begin{teo}
	\label{teo:teo1}
	Existe uma função $f : Q \to \mathbb{Z}$ e um inteiro $n$ tal que \begin{equation*}
	f(q_0) = 0 \wedge [(\forall q \in Q)(\forall e \in E), f(qe) \geq f(q) + t(e) \wedge f(q) \leq n]
	\end{equation*} se e somente se a fila do sistema modelado por $G$ tem sempre no máximo $n$ itens.
\end{teo}

\hl{INTRODUZIR PROVA} Provemos antes o lema que segue.

\begin{lem}
	\label{lem:lema1}
	Para toda função $f : Q \to \mathbb{Z}$, é válido que\begin{equation*}
	\begin{aligned}
	f(q_0) = 0 \wedge [(\forall q \in Q)(\forall e \in E), f(qe) \geq f(q) + t(e)]\\\Rightarrow \forall w \in L(G), c(w) \leq f(q_0w)
	\end{aligned}
	\end{equation*}
\end{lem}
\begin{proof}
A prova deste lema será por indução em $w$, tendo como base $$c(\varepsilon) \leq f(q_0\varepsilon)$$ o que é verdade, pois $c(\varepsilon) = 0$ e é pressuposto $f(q_0) = 0$.

Para o passo indutivo, temos de provar que, para todo evento $u \in E$, se $ wu$ é gerada por $G$, então $$c(wu) = c(w) + t(u) \leq f(q_0wu)$$ ou \begin{equation}
\label{eq:goallema1}
c(w) \leq f(q_0wu) - t(u)
\end{equation} é válido a partir da hipótese de indução.

Sabemos que, se $wu$ é gerada por $G$, então o prefixo $w$ também é. Diante disso \begin{equation}
\label{eq:hipindlema1}
c(w) \leq f(q_0w)
\end{equation} é obtido da hipótese de indução.

Com base na hipótese do lema, obtemos \begin{equation}
\label{eq:hiplema1}
f(q_0w) \leq f(q_0wu) - t(u)
\end{equation}

A partir das equações \ref{eq:hipindlema1} e \ref{eq:hiplema1}, verifica-se $$c(w) \leq f(q_0w) \leq f(q_0wu) - t(u)$$ demonstrando a equação \ref{eq:goallema1}, como queríamos.
\end{proof}

\hl{INTRODUZIR PROVA $\Rightarrow$ DO TEOREMA}

\begin{proof}
Aplicando o lema \ref{lem:lema1} na hipótese do teorema, obtemos $$\forall w \in L(G), c(w) \leq f(q_0w)$$

Da hipótese também temos $$f(q_0w) \leq n$$

Portanto $$c(w) \leq f(q_0w) \leq n$$ como desejávamos demonstrar.
\end{proof}

A demonstração do lema 1 e teorema 1 foram auxiliadas pelo Coq.

Na figura \ref{fig:ex1_teo1}, há um exemplo que ilustra um sistema de produtor e consumidor. Nela os eventos $p$ e $c$ são, respectivamente, a produção e consumo de item. Seja $f_1 : Q \to \mathbb{Z}$ uma função cuja imagem está rotulada nos nós da figura, indicando para cada estado o valor mapeado. Como $f_1$ respeita o teorema \ref{teo:teo1}, podemos concluir que o sistema modelado permite volume máximo de um item na fila.

\figuradoautor{Exemplo de aplicação do teorema \ref{teo:teo1} em um sistema que atende à propriedade}{
	\begin{tikzpicture}[shorten >=1pt,node distance=2cm,on grid,auto] 
	\node[state,initial] (q0) {$0$}; 
	\node[state] at (3,0) (q1) {$0$};
	\node[state] at (6,0) (q2) {$1$};
	\node[state] at (0,-3) (q3) {$0$};
	\node[state] at (3,-3) (q4) {$0$};
	\node[state] at (6,-3) (q5) {$1$};
	\draw
	(q0) edge[above] node{$p$} (q1)
	(q1) edge[above] node{$+$} (q2)
	(q2) edge[above,bend left=20] node{$-$} (q3)
	(q3) edge[below] node{$p$} (q4)
	(q4) edge[below] node{$+$} (q5)
	(q3) edge[left] node{$c$} (q0)
	(q4) edge[left] node{$c$} (q1)
	(q5) edge[left] node{$c$} (q2);
	\end{tikzpicture}
}{fig:ex1_teo1}

Sob outra perspectiva, a figura \ref{fig:ex2_teo1} apresenta um exemplo de AFD que modela um sistema que permite, notavelmente, adição e retirada de um número indefinido de itens da fila. Supondo que tal sistema atenda à propriedade da capacidade máxima, seja $f_2 : Q \to \mathbb{Z}$ qualquer função que mapeie cada estado a um número inteiro de forma que $f_2(q_0) = 0$ e $f_2(qe) \geq f_2(q) + t(e)$ para todo evento $e \in \{+,-\}$. É fácil notar que $f_2$ não existe, pois, como $q_1e = q_1$, podemos obter o absurdo: $f(q_1) \geq f(q_1) + t(e)$. Com isso, o teorema \ref{teo:teo1} conclui que o sistema não admite capacidade máxima.

\figuradoautor{Exemplo de sistema que não respeita a propriedade}{
	\begin{tikzpicture}[shorten >=1pt,node distance=2cm,on grid,auto] 
	\node[state,initial] (q0) {$q_0$};
	\node[state,initial] at (3,0) (q1) {$q_1$};
	\draw
	(q0) edge node{$+$} (q1)
	(q1) edge[loop above] node{$+$,$-$} (q1);
	\end{tikzpicture}
}{fig:ex2_teo1}

Além de provar que determinados sistemas de filas cumprem ou não com a especificação de uma fila com volume máximo, muitas vezes quer-se atestar a impossibilidade de efetuar operação de retirada de fila vazia. É sobre isso que trata a seção \ref{sec:vol_min}, a seguir.

\section{GARANTIA DE VOLUME MÍNIMO}
\label{sec:vol_min}

Todo sistema de filas especificado com base em \citeonline{cassandras} deve impedir a remoção de itens de filas quando estas forem vazias. Isso é em razão do caráter físico e não abstrato com que se configuram as filas por aquela ótica. Mesmo para filas abstratas, pode ser interessante que esta propriedade seja respeitada. A fim de garantir isso, o teorema \ref{teo:teo2} nos possibilita obter resposta à pergunta: a fila tem volume mínimo de zero itens?

\begin{teo}
	\label{teo:teo2}
	Existe uma função $f : Q \to \mathbb{Z}$ tal que \begin{equation*}
	f(q_0) = 0 \wedge [(\forall q \in Q)(\forall e \in E), f(qe) \leq f(q) + t(e) \wedge f(q) \geq 0]
	\end{equation*} se e somente se a fila do sistema modelado por $G$ nunca tem volume negativo.
\end{teo}

Considerando o autômato da figura \ref{fig:ex1_teo2}, verificamos que o sistema nela representado tem uma fila que respeita esta propriedade, embora não garanta capacidade máxima. Tomando a função $f_3 : Q \to \mathbb{Z}$ rotulada nos nós da figura, pode-se notar que ela respeita o teorema \ref{teo:teo2}, evidenciando que tal SED foi modelado de modo a não permitir o evento $-$ nos momentos em que a fila está vazia.

\figuradoautor{Exemplo de sistema que atende à propriedade}{
	\begin{tikzpicture}[shorten >=1pt,node distance=2cm,on grid,auto] 
	\node[state,initial] (q0) {$0$};
	\node[state] at (3,0) (q1) {$1$};
	\node[state] at (1.5,-1.5) (q2) {$0$};
	\draw
	(q0) edge node{$+$} (q1)
	(q1) edge node{$-$} (q2)
	(q2) edge node{$+$} (q0);
	\end{tikzpicture}
}{fig:ex1_teo2}

\figuradoautor{Exemplo de sistema que atende à propriedade}{
	\begin{tikzpicture}[shorten >=1pt,node distance=2cm,on grid,auto] 
	\node[state,initial] (q0) {$0$};
	\node[state] at (3,0) (q1) {$1$};
	\node[state] at (1.5,-1.5) (q2) {$0$};
	\draw
	(q0) edge node{$-$} (q1)
	(q1) edge node{$+$} (q2)
	(q2) edge node{$-$} (q0);
	\end{tikzpicture}
}{fig:ex1_teo2}

