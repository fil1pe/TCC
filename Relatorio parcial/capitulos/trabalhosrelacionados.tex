\chapter{Trabalhos relacionados}
\label{cap:relacionados}

\citeonline{doczkal} apresentam uma teoria construtiva de linguagens regulares em Coq na qual provam o teorema de Kleene, o lema do bombeamento, a unicidade do autômato determinístico mínimo, o teorema de Myhill-Nerode e propriedades sobre expressões regulares. No trabalho, os autores definem as expressões regulares de maneira indutiva no Coq e formalizam autômatos finitos determinísticos (\acs{AFD}s) e não determinísticos (\acs{AFND}S) como dois tipos \texttt{record}. Para garantir que os tipos dos estados e símbolos dos autômatos sejam finitos, eles utilizam uma extensão do Coq e o tipo \texttt{char} respectivamente. O trabalho resultou em cerca de 1400 linhas de código e 170 lemas.

\citeonline{braibant} propõem uma tática para resolver equações e inequações em álgebras de Kleene, uma porção de relações binárias decidíveis não triviais constituída pelas constantes e operadores das linguagens regulares. Para verificar se duas expressões regulares expressam a mesma linguagem, os autores desenvolveram um algoritmo que constrói um AFND com transições $\varepsilon$ para cada expressão, converte-os para AFDs e verifica a equivalência entre eles. O desenvolvimento foi feito no Coq, bem como a prova de que ele é correto. O trabalho originou um algoritmo de 10000 linhas eficiente, se comparado seu desempenho de tempo com o dos outros algoritmos.

Outro trabalho relacionado que implementa autômatos usando tipos \texttt{record} do Coq é o de \citeonline{athalye}, que versa sobre autômatos de entrada e saída: \textit{IO automata}. Tais autômatos são, como o autor define, modelos de componentes de sistemas distribuídos, servindo de formalismos matemáticos para garantir o corretismo de projetos e implementações de algoritmos distribuídos e protocolos. O trabalho assemelha-se a este no sentido de modelar possíveis sistemas a eventos discretos (\acs{SED}s) usando autômatos e de provar características sobre autômatos no assistente Coq. O autor destaca a explosão combinatorial decorrente do \textit{model checking}, o que faz essa técnica ser inutilizável na verificação de sistemas reais.

Sistemas híbridos são sistemas orientados a eventos e tempo. Nesse contexto, \citeonline{tveretina} sugere decompor o espaço dos estados em uma grade retangular, com o objetivo de reduzir custos na verificação de segurança. A tarefa consiste em buscar resposta à pergunta: ``há como chegar a um estado não seguro?''. Para tanto, a autora implementou um modelo de autômato híbrido em Coq e utilizou-o para modelar um controlador de portão de passagem de trem. Haja vista a semelhança entre SEDs e sistemas híbridos, esse trabalho relacionado evidencia a relevância do Coq na verificação de segurança para sistemas variados.

Os trabalhos supracitados compõem uma revisão de literatura acerca da modelagem e verificação de SEDs mediante o assistente de provas Coq. Tendo em vista a relação entre AFDs e SEDs, entre modelos de autômatos e entre linguagens regulares e AFDs, os quatro trabalhos relacionados correlacionam-se com o presente trabalho de conclusão de curso. Não se detectou publicação a respeito da especificação e prova de propriedades de sistemas de filas, o que garante ao presente trabalho relativa originalidade.