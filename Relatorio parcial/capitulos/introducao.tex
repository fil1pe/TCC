\chapter{Introdução}

Os autômatos são modelos de máquinas abstratas muito importantes para os estudos da computação teórica. Em particular, os autômatos finitos determinísticos (\acs{AFD}s) reconhecem as linguagens regulares, aplicadas na ciência da computação e tecnologia da informação \cite{hopcroft}. Sob outra perspectiva, essa classe de autômatos tem sido utilizada para descrever sistemas do mundo hodierno na automação de indústrias e tecnologias, já que o advento dos sistemas a eventos discretos (\acs{SED}s) impôs um novo paradigma, contrário aos sistemas orientados pelo tempo, que são, em geral, muito bem descritos pelos modelos físicos clássicos e modernos \cite{cassandras}.

Com a finalidade de descrever o funcionamento dos sistemas que são orientados a eventos discretos e assíncronos, eles podem ser modelados por formalismos da matemática discreta, entre os quais se pode citar os autômatos determinísticos, redes de Petri, \textit{timed models} e cadeias de Markov. São exemplos de SEDs que podem ser representados por esses modelos os sistemas de filas, computacionais, de comunicação, manufatura, tráfego, banco de dados e telefonia. Os sistemas de filas, geralmente, envolvem a sincronização de processos manipulando filas ou \textit{buffers}, problema presente na automação industrial. É possível modelar diferentes SEDs com as características dos sistemas de filas a fim de representar plantas industriais com recursos compartilhados entre suas partes. Os AFDs são amplamente usados nesse sentido, já que reduzem o tempo do projeto desses sistemas, permitem formalizá-los e torna mais fácil a manutenção deles \cite{rosso1}.

A obtenção de propriedades sobre AFDs específicos pode não ser uma tarefa trivial, e sua formalização, a prova, costuma mostrar-se ainda mais difícil, pois a intuição, muitas vezes, é o único meio utilizado para a garantia das propriedades. Outra técnica utilizada com esse fim é o \textit{model checking}, no qual os instrumentos de verificação -- \textit{model checkers} -- operam enumerando exaustivamente o espaço dos estados de um modelo. O método é limitante, porque resulta em uma explosão combinatorial, tornando impraticável a verificação de modelos de sistemas reais desse modo \cite{athalye}. Por isso, o emprego de assistentes interativos de provas matemáticas é útil, eles possibilitam a formulação e demonstração das propriedades. Tais assistentes reúnem meios formais e computacionais que propiciam a verificação dos passos da demonstração, auferindo rigor. Entre os assistentes de provas, o Coq destaca-se por ter comunidade extensa, ter uma linguagem funcional de ordem superior e permitir a construção de tipos dependentes. Outras qualidades desse assistente são a possibilidade de programar táticas de provas, desenvolver e carregar módulos separadamente sem a necessidade de verificá-los novamente e definir notações para melhorar a visualização das definições e provas \cite{manualcoq}. O Coq tem se mostrado uma ferramente robusta e bastante aceita pela comunidade acadêmica. O teorema das quatro cores, segundo o qual qualquer mapa pode ser colorido com quatro cores de modo que as regiões adjacentes tenham cores diferentes, foi provado em Coq por \citeonline{4cores}, resultando em uma prova mais rigorosa em relação às demais demonstrações para o mesmo teorema. Embora o computador tenha sido empregado nas outras provas, essa última destaca-se por dispensar a necessidade de confiar nos programas verificadores de casos particulares e ser expressa na linguagem formal do Coq, à qual se atribuiu maior confiabilidade. Essas e outras características do Coq motivaram a escolha dele para a assistência das provas que este trabalho externa.

Os problemas envolvendo filas em processos concorrentes são muito estudados na ciência da computação e dotados de relevância no que tange aos sistemas de produção industrial, onde a falta de sincronização acarreta eventualmente outros problemas. Se um SED é corretamente modelado por AFDs, deve ser viável identificar nos modelos as eventuais falhas decorrentes da concorrência por recursos distribuídos em filas; senão, apontar os defeitos da representação. Por esse motivo, confere-se valor e importância ao presente estudo.

O objetivo geral deste trabalho de conclusão de curso consiste em provar, mediante o assistente de provas Coq, propriedades sobre sistemas da automação industrial modelados por AFDs. Os objetivos específicos são: \begin{enumerate}
	\item introduzir os principais conceitos relativos aos SEDs;
	\item apresentar a classe de sistemas de filas;
	\item especificar propriedades desejadas para essa classe;
	\item empregar o assistente Coq na prova de teoremas referentes a essas propriedades;
	\item demonstrar e explanar tais teoremas e exemplificar sua aplicação.
\end{enumerate}

Com a finalidade de atingir os objetivos supracitados, o presente trabalho é estruturados destarte. Primeiramente, o Capítulo \ref{cap:provas} introduz o assistente de provas Coq. No Capítulo \ref{cap:seds}, são introduzidos os SEDs, assim como sua modelagem por AFDs. Já no Capítulo \ref{cap:relacionados}, apresentam-se os trabalhos relacionados a este estudo. O capítulo seguinte, \ref{cap:filas}, apresenta os sistemas de filas, problemas relacionados e as propriedades mais importantes acerca desses SEDs. Em seguida, o Capítulo \ref{cap:propriedades} demonstra como constatar se sistemas de filas atendem às propriedades. Por fim, são expressas as considerações finais e, depois, as referências bibliográficas.